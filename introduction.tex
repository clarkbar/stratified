%!TEX root = rootfile.tex
% chktex-file 3
% chktex-file 8
% chktex-file 12
% chktex-file 24
% chktex-file 42

Our work here stems from a general principle: if $ X $ is a (nice) geometric object (a topological space, a variety, \emph{etc}.), then a suitable class $ A $ of sheaves on $ X $ determines and is determined by a certain \enquote{homotopy type} $ X_A $ attached to $ X $.
The larger the class $ A $, the finer we can expect the homotopy type $ X_A $ to be.

In this introduction, our aim is to examine this principle in a four relatively familiar situations:
\begin{itemize}
	\item when $ X $ is a topological space, and $ A $ is the family of all sheaves;
	\item when $ X $ is a topological space, and $ A $ is the family of constant sheaves;
	\item when $ X $ is a topological space, and $ A $ is the family of locally constant sheaves;
	\item when $ X $ is a scheme, and $ A $ is the family of locally constant étale sheaves. 
\end{itemize}
In these circumstances, the object $ X_A $ is a homotopy type (or piece thereof) in more or less the usual sense.

In the body of this text, we will study two further situations:
\begin{itemize}
	\item when $ X $ is a stratified topological space, and $ A $ is the family of constructible sheaves;
	\item when $ X $ is a scheme, and $ A $ is the family of constructible sheaves.
\end{itemize}
In these situations, the object $ X_A $ will be a \emph{stratified homotopy type}.
We will see that this is a dramatically different object, but it comes with much finer information about the structure of $ X $.

\subsection{Notes on sobriety}%
\label{sub:notes_on_sobriety}

The most elementary illustration of the principle above is the consideration of topological spaces $ X $ that can be recovered from their entire categories of sheaves.
These are called the \emph{sober} topological spaces.

\begin{dfn}
	Let $ X $ be a topological space.
	A closed subset $ Z \subseteq X $ is \defn{irreducible} if and only if it is nonempty and if, for any closed subsets $ Z_1, Z_2 \subseteq X $ such that $ Z = Z_1 \cup Z_2 $, either $ Z = Z_1 $ or $ Z = Z_2 $.
	Dually, an open subset $ U \subset X $ is \defn{irreducibly open} if and only if it is proper and if, for any open subsets $ U_1, U_2 \subseteq X $ such that $ U_1 \cap U_2 \subseteq U $, either $ U_1 \subseteq U $ or $ U_2 \subseteq U $.
\end{dfn}

\begin{exm}
	If $ X $ is a topological space with a point $ x \in X $, then the closure $ \overline{x} $ is irreducible.
	In this case, one says that $ x $ is a \defn{generic point} of $ X $.
	The union of all the open subsets that do not contain $ x $ is irreducibly open.
\end{exm}

\begin{dfn}
	Let $ X $ be a topological space, and let $ I(X) $ be the set of irreducible closed subsets of $ X $.
	Then we say that $ X $ is \defn{sober} if the map $ X \to I(X) $ given by $ x \mapsto \overline{x} $ is a bijection;
	that is, $ X $ is sober if and only if every irreducible closed subset of $ X $ has a unique generic point.
\end{dfn}

\begin{nul}
	Sober topological spaces are always Kolmogoroff
	(\textit{i.e.}, any distinct points are topologically distinguishable).
	Indeed, Kolmogoroff topological spaces are exactly those with the property that the map $ x \mapsto \overline{x} $ is an injection.
\end{nul}

\begin{exm}
	Any Hausdorff space is sober.
\end{exm}

\begin{exm}
	The Zariski topology of any scheme is sober.
	In particular, not all sober topological spaces are T1.
\end{exm}

\begin{exm}
	The set $ \NN $ with the cofinite topology is T1 but not sober.
\end{exm}

\begin{cnstr}
	Let $ X $ be a topological space, and let $ I(X) $ be the set of irreducible closed subsets thereof.
	Write $ \gamma $ for the map $ x \mapsto \overline{x} $.
	We topologise $ I(X) $ with the finest topology such that $ \gamma_X $ is continuous;
	in other words, a subset $U \subseteq I(X) $ is open if and only if $ \gamma_X^{-1} U $ is open in $ X $.
	Equivalently, $ U $ is open if and only if there exists an open subset $ V \subseteq X $ such that $ U = \{ F \in I(X) : F \cap V \neq \varnothing \} $.

	Thus $ \gamma $ induces a bijection between the opens of $ I(X) $ and the opens of $ X $, and $ I(X) $ is sober.
	
	The assignment $ X \mapsto I(X) $ along with the natural transformation $ \gamma $ together define a left adjoint to the forgetful functor from sober topological spaces to all topological spaces.
	Thus one may \defn{sober up} any topological space $ X $.
\end{cnstr}

Now let us demonstrate that a sober space is completely controlled by its lattice of open sets.

\begin{ntn}
	Let $ X $ be a topological space.
	Then we write $ \Omega(X) $ for the poset of open subsets of $ X $;
	this is a complete lattice.
	If $ Y $ is also a topological space, then we write $ \Map^{\textit{cl}}(\Omega(Y), \Omega(X)) $ for the set of \emph{complete lattice homomorphisms} -- monotonic maps $ \Omega(Y) \to \Omega(X) $ that preserve finite intersections and arbitrary unions.
\end{ntn}

\begin{prp}
	Let $ X $ and $ Y $ be sober topological spaces;
	Then the assignment $ f \mapsto f^{-1} $ is a bijection
	\[
		\Map(X, Y) \equivalence \Map^{\textit{cl}}(\Omega(Y), \Omega(X)) \period
	\]
\end{prp}

\begin{proof}
	Let $ F \colon \Omega(Y) \to \Omega(X) $ be a monotonic map that preserves finite intersections and arbitrary unions.
	Now let us define a map $ f \colon X \to Y $: let $ x \in X $.
	For each of $ X $ and $ Y $, the points are in bijection with the irreducible closed subsets, which are in turn in bijection with the irreducibly open subsets.
	Now the map $ F $ induces a map from the irreducibly open subsets of $ X $ to those of $ Y $:
	this map carries an irreducible open $ U \subseteq X $ to the largest open $ V \subseteq Y $ such that $ F(V) \subseteq U $.
	Following these bijections, we obtain a unique map $ f $ with the property that $ f^{-1} = F $.
	
	To be explicit, let $ V \in \Omega(Y) $ be the largest open subset such that $ x \notin F(V) $.
	This subset is irreducibly open, and so its complement is an irreducible closed.
	The value $ f(x) $ is the unique generic point of $ Y \smallsetminus V $.
\end{proof}

This provides a fully faithful embedding of sober topological spaces into complete lattices.
We now show that this provides a similar fully faithful embedding of sober topological spaces into topoi.

\begin{ntn}
	Let $ X $ be a topological space.
	We shall write $ \widetilde{X}_{\leq 0} $ for the topos of sheaves of sets on $ X $.
	(This notation may seem a little strange, but soon we will have to contend with sheaves of \emph{spaces} on $ X $, and we are reserving this notation for that notion.)

	For any two topoi $ \XX $ and $ \YY $, we write $ \Funlowerstar( \XX, \YY ) $ for the category whose objects are geometric morphisms $ f_{\ast} \colon \XX \to \YY $ and whose morphisms are natural transformations $ f_{\ast} \to g_{\ast} $.
\end{ntn}

\begin{nul}
	If $ X $ is a topological space, then we can recover $ \Omega(X) $ from $ \widetilde{X}_{\leq 0} $ as the poset of subobjects of the terminal object $1_{\widetilde{X}_{\leq 0}}$.
	Now we can make the following little observation: if $ X $ and $ Y $ are topological spaces, then the left adjoint of a geometric morphism $ f_{\ast} \colon \widetilde{X}_{\leq 0} \to \widetilde{Y}_{\leq 0} $ restricts to a complete lattice homomorphism $ \Omega(X) \to \Omega(Y) $.
	This is an inverse to the obvious functor $ \Map^{\textit{cl}}(\Omega(Y), \Omega(X)) \to \Funlowerstar(\widetilde{X}_{\leq 0}, \widetilde{Y}_{\leq 0})$.

	In particular, please observe that the geometric morphisms $ \widetilde{X}_{\leq 0} \to \widetilde{Y}_{\leq 0} $ form a discrete category.
\end{nul}

\begin{cor}
	Let $ X $ and $ Y $ be sober topological spaces.
	Then the assignment $ f \mapsto f_{\ast} $ is an equivalence of categories
	\[
		\Map(X, Y) \equivalence \Funlowerstar(\widetilde{X}_{\leq 0}, \widetilde{Y}_{\leq 0}) \period
	\]
\end{cor}

\subsection*{Connectedness and constant sheaves}

Let us begin by understanding the nature of connectedness.

\begin{ntn}
	We well begin by contemplating sheaves of \emph{sets} on topological spaces.
	For any topological space $ X $, let $ \widetilde{X}_{\leq 0} $ be the category of sheaves of sets on $ X $.
	(Eventually, we shall have to consider sheaves of \emph{spaces} on $ X $, and we are saving the notation $ \widetilde{X} $ for that.)
\end{ntn}

\begin{dfn}
	The \defn{constant sheaf} at a set $ S $ on a topological space $ X $ is the sheafification of the constant presheaf $ U \mapsto S $.
\end{dfn}

\begin{nul}
	The formation of the constant sheaf defines a left exact left adjoint
	\[
		\Gammaupperstar_X \colon \Set \to \widetilde{X}_{\leq 0} \period
	\]
	Its right adjoint $ \Gamma_{X, \ast} $ is the formation of global sections $ F \mapsto F(X) $.
\end{nul}

\begin{nul}
	For any topological space $ X $, any set $ S $, and any point $ x \in X $, the stalk of the constant sheaf $ \Gammaupperstar_X (S) $ at $ X $ is canonically isomorphic to $ S $.

	Indeed, $ x^{\ast} \Gammaupperstar_X $ is a left exact left adjoint from $ \Set $ to itself;
	such a functor is isomorphic in a unique fashion to the identity.
\end{nul}

\begin{exm}
	Constant sheaves are not constant as presheaves.
	Indeed, let $ X $ be the discrete space $ \{ 0, 1 \} $, and let $ S $ be any set.
	Then the constant sheaf at $ S $ on $ X $ certainly has the property that its values on $ \{ 0 \} $ and $ \{ 1 \} $ are each the set $ S $, but now the sheaf condition requires that the global sections are given by
	\[
		\Gamma_{X, \ast} \Gammaupperstar_X (S) = \Gammaupperstar_X (S) \{0,1\} \cong \Gammaupperstar_X (S) \{ 0 \} \times \Gammaupperstar_X (S) \{ 1 \} \cong S \times S \period
	\]
	If $ S $ has at least two elements, then it follows that $ \Gammaupperstar_X (S) $ is not constant as a presheaf.
\end{exm}

The previous example does illustrate a general principle:

\begin{lem}
	Let $ X $ be a discrete topological space, and let $ S $ be a set.
	Then $ \Gamma_{X, \ast} \Gammaupperstar_X ( S ) \cong \Map(X, S ) $. 
\end{lem}

If we want to understand various constant sheaves, we can do so by coming to grips with the various functor $ \Gamma_{X, \ast} \Gammaupperstar_X \colon \Set \to \Set $ as $ X $ varies.
The first thing we can notice is that this functor is a left exact accessible functor.

\begin{dfn}
	A \defn{proöbject} of a category $ C $ with all finite limits is a left exact accessible functor $ C \to \Set $.
	The category $ \Pro(C) $ of proöbjects is the \emph{opposite} of the full subcategory of $ \Fun(C, \Set) $ spanned by the left exact accessible functors.
\end{dfn}

\begin{exm}
	If $ C $ is a category with all finite limits, then the Yoneda embedding provides a fully faithful functor $ \yo \colon C \inclusion \Pro(C) $.
	Explicitly, if $ X $ is an object of a category $ C $, then the proöbject it defines is $ \Map(X, -) $.
\end{exm}

\begin{exm}
	As a matter of terminology, we call the opposite $ A^{\op} $ of a filtered category an \defn{inverse} category;
	we call a diagram indexed by an inverse category an \defn{inverse system};
	and we call a limit of an inverse system an \defn{inverse limit}.

	If $ X \colon A^{\op} \to C $ is an inverse system in a category $ C $ with all finite limits, then the limit $ \lim_{\alpha \in A^{\op}} X_{\alpha} $ formed in $ \Fun(C, Set)^{\op} $ is a proöbject;
	this is the proöbject $ Y \mapsto \colim_{\alpha \in A} \Map(X_{\alpha}, Y) $.
	Furthermore, \emph{every} proöbject of $ C $ can be formed in such a manner.

	Now if $ X $ and $ Y $ are two proöbjects that are exhibited as limits of inverse systems in this sense, then one has
	\[
		\Map_{\Pro(C)}(X, Y) \cong \lim_{\beta \in B^{\op}} \colim_{\alpha \in A} \Map_C(X, Y) \period
	\]
\end{exm}

\begin{exm}
	Let $ X $ be a topological space.
	We obtain a proset $ \pi^{\toptextit}_0(X) \coloneq \Gamma_{X, \ast}\Gammaupperstar_X $.
	This defines a functor from topological spaces to prosets.

	More generally, this is a functor from topoi to prosets:
	this carries a topos $ \XX $ to the composite
	\[
		\pi_0(\XX) \coloneq \Gamma_{\XX, \ast} \Gammaupperstar_{\XX} \colon \Set \to \XX \to \Set \period
	\]
\end{exm}

Our claim is that the proset $ \pi_0^{\toptextit}(X) $ is closely related to -- and even identifiable with -- the set $ \pi_0(X) $.
For a relatively nice class of topological spaces, $ \pi_0(X) $ has a simple universal property.

\begin{nul}
	Let us consider the category $ \Top^{\textit{ng}} $ of \defn{numerically generated} topological spaces -- these are topological spaces $ X $ with the property that a subset $ U \subseteq X $ is open if and only if, for any continuous map $ \gamma \colon \RR^n \to X $, the set $ gamma^{-1} U $ is open.
	
	Of course any discrete space is numerically generated, so the assignment $ S \mapsto S^{\disc} $ is a functor $ \Set \to \Top^{\textit{ng}} $.
	This functor has a left adjoint, $ \pi_0 \colon \Top^{\textit{ng}} $;
	in other words, for any numerically generated topological space $ X $, the continuous maps $ X \to S^{\disc} $ are in bijective correspondence with the maps $ \pi_0(X) \to S $.
\end{nul}

For any \defn{numerically generated} topological space $ X $ 

\begin{dfn}
	A morphism of topoi $ \plowerstar \colon \XX \to \YY $ is \defn{étale} if and only if the left adjoint $ \pupperstar $ admits a further left adjoint $ \plowershriek $ that identifies $ \XX $ with the overcategory $ \YY_{/\plowershriek(1_{\XX})} $.
\end{dfn}

\begin{cnstr}
	Let $ \XX $ and $ \YY $ be topoi, and let $ f_{\ast} \colon \XX \to \YY $ be a geometric morphism.
	Let $ U \in \YY $ be an object, and let $ \alpha \colon 1_{\XX} \to f^{\ast} U $ be a morphism of $ \XX $.
	Define a functor $ F^{\ast} \colon \YY_{/U} \to \XX $ that carries an object $ [ V \to U ] $ of $ \YY_{/U} $ to the object $ 1_{\XX} \times_{f^{\ast} U} f^{\ast} V $;
	this functor admits a right adjoint $ F_{\ast} $, which is a morphism of topoi.
	If we write $ \plowerstar \colon \YY_{/U} \to \YY $ for the canonical étale morphism, then we have an isomorphism $ \fupperstar \cong F^{\ast} \circ \pupperstar $, and
	it is a tedious but routine check to confirm that this defines a functor
	\[
		\Map_{\XX}(1_{\XX}, \fupperstar U) \to \Fun_{\ast,/\XX}(\XX, \YY_{/U}) \period
	\]
\end{cnstr}

\subsection*{Monodromy representations}
\begin{dfn}
	Let $ X $ be a topological space.

	A  \defn{locally constant sheaf} of sets $ L $ is a sheaf for which there exists an open cover $ \{ U_{\alpha} \}_{\alpha \in A} $ such that for any $ \alpha \in A $, the sheaf $ L | U_{\alpha} $ is constant on $ U_{\alpha} $.
	We will call such an open cover a \defn{trivialising} open cover of $ X $ for $ L $.

	A \defn{local system} is a locally constant sheaf for which there is a finite trivialising open cover.
\end{dfn}

\begin{nul}
	Let $ X $ be a topological space, and let $ L $ be a locally constant sheaf on $ X $. 
	Any refinement of a trivialising open cover of $ X $ for $ L $ is again trivialising.
	Note also that if $ X $ is quasicompact, then any locally constant sheaf is a local system.
\end{nul}

\begin{exm}
	Consider the interval $ I \coloneq [0, 1] $.
	Let $ L $ be a local system on $ I $.
	A finite trivialising open cover can be refined to a finite trivialising open cover $ \{ U_1, \dots, U_n \} $ in which each $ U_{i} $ is an interval.

	Now let $ U_{i} $ and $ U_{j} $ be two of these intervals;
	their intersection is either empty or else an inverval.
	Assume that their intersection is in fact nonempty;
	then their union is again an interval.
	Let $ S $ be a set such that $ L | U_{i} $ is constant at $ S $.
	Since $ U_{i} $ and $ U_{j} $ intersect, it follows that $ L | U_{j} $ is constant at $ S $ as well.
	The sheaf condition gives an equaliser
	\[
		L(U_{i} \cup U_{j}) \to S \times S \parallelto S \times S \times S \times S \comma
	\]
	where the top arrow is $ (s, t) \mapsto (s, s, t, t) $ and the bottom arrom is $ (s, t) \mapsto (s, t, s, t) $ which now implies that $ L(U_{i} \cup U_{j}) \cong S $ in a way that is compatible with the restrictions to $ U_{i} $ and $ U_{j} $.
	Thus we obtain a morphism $ \eta $ from the constant sheaf at $ S $ on $ U_{i} \cup U_{j} $ to the restriction $ L | (U_{i} \cup U_{j}) $
	that restricts to an isomorphism on $ U_{i} $ and $ U_{j} $.
	Thus it follows that $ \eta $ is an isomorphism.

	Thus any finite trivialising open cover consisting of $ n > 1$ intervals can be replaced by a finite trivialising open cover consisting of $ n - 1 $ intervals.
	By induction, it follows that $ I $ itself is a trivialising open cover, whence $ L $ is a constant sheaf.

	We thus conclude that any local system on the interval $ I $ is in fact constant!
\end{exm}

In the course of this discussion, we encountered some facts that will be useful to us more generally.

\begin{lem}
	Let $ X $ be a topological space, and let $ F $ be a sheaf of sets on $ X $.
	Then if $ U $ and $ V $ are open sets such that $ F | U $ and $ F | V $ are constant, then 
\end{lem}




